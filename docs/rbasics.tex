\documentclass[]{tufte-book}

% ams
\usepackage{amssymb,amsmath}

\usepackage{ifxetex,ifluatex}
\usepackage{fixltx2e} % provides \textsubscript
\ifnum 0\ifxetex 1\fi\ifluatex 1\fi=0 % if pdftex
  \usepackage[T1]{fontenc}
  \usepackage[utf8]{inputenc}
\else % if luatex or xelatex
  \makeatletter
  \@ifpackageloaded{fontspec}{}{\usepackage{fontspec}}
  \makeatother
  \defaultfontfeatures{Ligatures=TeX,Scale=MatchLowercase}
  \makeatletter
  \@ifpackageloaded{soul}{
     \renewcommand\allcapsspacing[1]{{\addfontfeature{LetterSpace=15}#1}}
     \renewcommand\smallcapsspacing[1]{{\addfontfeature{LetterSpace=10}#1}}
   }{}
  \makeatother
\fi

% graphix
\usepackage{graphicx}
\setkeys{Gin}{width=\linewidth,totalheight=\textheight,keepaspectratio}

% booktabs
\usepackage{booktabs}

% url
\usepackage{url}

% hyperref
\usepackage{hyperref}

% units.
\usepackage{units}


\setcounter{secnumdepth}{2}

% citations
\usepackage{natbib}
\bibliographystyle{apalike}
%\renewcommand{\bibsection}{\chapter*{References}}

% pandoc syntax highlighting
\usepackage{color}
\usepackage{fancyvrb}
\newcommand{\VerbBar}{|}
\newcommand{\VERB}{\Verb[commandchars=\\\{\}]}
\DefineVerbatimEnvironment{Highlighting}{Verbatim}{commandchars=\\\{\}}
% Add ',fontsize=\small' for more characters per line
\usepackage{framed}
\definecolor{shadecolor}{RGB}{248,248,248}
\newenvironment{Shaded}{\begin{snugshade}}{\end{snugshade}}
\newcommand{\AlertTok}[1]{\textcolor[rgb]{0.94,0.16,0.16}{#1}}
\newcommand{\AnnotationTok}[1]{\textcolor[rgb]{0.56,0.35,0.01}{\textbf{\textit{#1}}}}
\newcommand{\AttributeTok}[1]{\textcolor[rgb]{0.77,0.63,0.00}{#1}}
\newcommand{\BaseNTok}[1]{\textcolor[rgb]{0.00,0.00,0.81}{#1}}
\newcommand{\BuiltInTok}[1]{#1}
\newcommand{\CharTok}[1]{\textcolor[rgb]{0.31,0.60,0.02}{#1}}
\newcommand{\CommentTok}[1]{\textcolor[rgb]{0.56,0.35,0.01}{\textit{#1}}}
\newcommand{\CommentVarTok}[1]{\textcolor[rgb]{0.56,0.35,0.01}{\textbf{\textit{#1}}}}
\newcommand{\ConstantTok}[1]{\textcolor[rgb]{0.00,0.00,0.00}{#1}}
\newcommand{\ControlFlowTok}[1]{\textcolor[rgb]{0.13,0.29,0.53}{\textbf{#1}}}
\newcommand{\DataTypeTok}[1]{\textcolor[rgb]{0.13,0.29,0.53}{#1}}
\newcommand{\DecValTok}[1]{\textcolor[rgb]{0.00,0.00,0.81}{#1}}
\newcommand{\DocumentationTok}[1]{\textcolor[rgb]{0.56,0.35,0.01}{\textbf{\textit{#1}}}}
\newcommand{\ErrorTok}[1]{\textcolor[rgb]{0.64,0.00,0.00}{\textbf{#1}}}
\newcommand{\ExtensionTok}[1]{#1}
\newcommand{\FloatTok}[1]{\textcolor[rgb]{0.00,0.00,0.81}{#1}}
\newcommand{\FunctionTok}[1]{\textcolor[rgb]{0.00,0.00,0.00}{#1}}
\newcommand{\ImportTok}[1]{#1}
\newcommand{\InformationTok}[1]{\textcolor[rgb]{0.56,0.35,0.01}{\textbf{\textit{#1}}}}
\newcommand{\KeywordTok}[1]{\textcolor[rgb]{0.13,0.29,0.53}{\textbf{#1}}}
\newcommand{\NormalTok}[1]{#1}
\newcommand{\OperatorTok}[1]{\textcolor[rgb]{0.81,0.36,0.00}{\textbf{#1}}}
\newcommand{\OtherTok}[1]{\textcolor[rgb]{0.56,0.35,0.01}{#1}}
\newcommand{\PreprocessorTok}[1]{\textcolor[rgb]{0.56,0.35,0.01}{\textit{#1}}}
\newcommand{\RegionMarkerTok}[1]{#1}
\newcommand{\SpecialCharTok}[1]{\textcolor[rgb]{0.00,0.00,0.00}{#1}}
\newcommand{\SpecialStringTok}[1]{\textcolor[rgb]{0.31,0.60,0.02}{#1}}
\newcommand{\StringTok}[1]{\textcolor[rgb]{0.31,0.60,0.02}{#1}}
\newcommand{\VariableTok}[1]{\textcolor[rgb]{0.00,0.00,0.00}{#1}}
\newcommand{\VerbatimStringTok}[1]{\textcolor[rgb]{0.31,0.60,0.02}{#1}}
\newcommand{\WarningTok}[1]{\textcolor[rgb]{0.56,0.35,0.01}{\textbf{\textit{#1}}}}

% longtable
\usepackage{longtable,booktabs}

% multiplecol
\usepackage{multicol}

% strikeout
\usepackage[normalem]{ulem}

% morefloats
\usepackage{morefloats}

% force floats added by CII
\usepackage{float}
\floatplacement{figure}{H}


% tightlist macro required by pandoc >= 1.14
\providecommand{\tightlist}{%
  \setlength{\itemsep}{0pt}\setlength{\parskip}{0pt}}

% title / author / date
\title{Getting Used to R, RStudio, and R Markdown}

\author{Chester Ismay and Patrick C. Kennedy}
\date{2019-11-12}

\usepackage{booktabs}
\usepackage{longtable}
\usepackage{framed,color}
\usepackage{float}
\floatplacement{figure}{H}
\usepackage[parfill]{parskip}
\definecolor{shadecolor}{RGB}{248,248,248}

\ifxetex
  \usepackage{letltxmacro}
  \setlength{\XeTeXLinkMargin}{1pt}
  \LetLtxMacro\SavedIncludeGraphics\includegraphics
  \def\includegraphics#1#{% #1 catches optional stuff (star/opt. arg.)
    \IncludeGraphicsAux{#1}%
  }%
  \newcommand*{\IncludeGraphicsAux}[2]{%
    \XeTeXLinkBox{%
      \SavedIncludeGraphics#1{#2}%
    }%
  }%
\fi

%% Need to clean up
\newenvironment{rmdblock}[1]
  {\begin{shaded*}
  \begin{itemize}
  \renewcommand{\labelitemi}{
    \raisebox{-.7\height}[0pt][0pt]{
  %    {\setkeys{Gin}{width=3em,keepaspectratio}\includegraphics{images/#1}}
    }
  }
  \item
  }
  {
  \end{itemize}
  \end{shaded*}
  }
%% Probably can be omitted
\newenvironment{rmdnote}
  {\begin{rmdblock}{note}}
  {\end{rmdblock}}
\newenvironment{rmdcaution}
  {\begin{rmdblock}{caution}}
  {\end{rmdblock}}
\newenvironment{rmdimportant}
  {\begin{rmdblock}{important}}
  {\end{rmdblock}}
\newenvironment{rmdtip}
  {\begin{rmdblock}{tip}}
  {\end{rmdblock}}
\newenvironment{rmdwarning}
  {\begin{rmdblock}{warning}}
  {\end{rmdblock}}
\newenvironment{learncheck}
  {\begin{rmdblock}{warning}}
  {\end{rmdblock}}
\newenvironment{review}
  {\begin{rmdblock}{warning}}
  {\end{rmdblock}}

% To tweak tufte layout
\geometry{
  left=0.8in, % left margin
  textwidth=35pc, % main text block
  marginparsep=1pc, % gutter between main text block and margin notes
  marginparwidth=8pc % width of margin notes
}

\begin{document}

%\let\allcaps=\relax
\maketitle



{
\setcounter{tocdepth}{1}
\tableofcontents
}

\hypertarget{intro}{%
\chapter{Introduction}\label{intro}}

This book was written to give people who are new to \href{https://www.r-project.org}{R}, \href{https://www.rstudio.com}{RStudio}, and \href{http://rmarkdown.rstudio.com}{R Markdown} the tools they need to begin making their own research reproducible. R is an open-source programming language that has seen its popularity grow tremendously in recent years, with developers adding new functionality via packages on a daily basis. RStudio is a graphical development environment that makes it easier to write and view the results of R code, and R Markdown provides an easy way to produce rich, fully-documented, reproducible analyses.

Screenshots and screencasts (with no audio) are used throughout to illustrate key concepts, but if you need further clarification on these or any other aspect, please create a \href{https://github.com/ismayc/rbasics-book/issues}{GitHub issue} or \href{mailto:chester.ismay@gmail.com}{email me} with a reference to the area where more guidance is necessary. \href{https://code.tutsplus.com/courses/git-basics-github-pull-requests}{Pull requests} on GitHub for typos or improvements are also welcome, and you can easily do so by clicking on the Edit button near Search at the top of the HTML version of the book.

In the HTML version, you can download the book as a PDF by clicking on the PDF button in the toolbar at the top of the page. Given the heavy use of screencasts, HTML is the recommended format for most readers, but the PDF is available for those who need it. Links to the different YouTube videos directly found in the HTML version are provided in the PDF version. You can also download the video files \href{https://github.com/ismayc/rbasics-book/tree/master/mp4}{directly}. Note that no audio is attached to these screencast videos.

This book will evolve and be updated as needed based on reader feedback. You can see when the book was last updated below.

I strongly recommend that you use R version 3.3.0 or higher, RStudio Desktop version 1.0 or higher, and \texttt{rmarkdown} R package version 1.0 or higher. This will ensure that your setup matches the screenshots and recordings, making it easier to follow along. Additionally, you may find that videos don't load sometimes. I haven't had any problems using Google Chrome and recommend that as your browser to view this book if you have trouble with other browsers.

This book was last updated by chester.ismay on Tuesday, November 12, 2019 16:08:10 MST.

\hypertarget{whyR}{%
\chapter{Why R?}\label{whyR}}

If you are new to R and programming, you may be intimidated by the idea of writing code. You probably aren't used to having to type commands to tell the computer what to do. You may be more comfortable using drop-down menus and other graphical user interfaces that allow you to pick what you'd like to do. So, why are so many companies, colleges, universities, and individuals of all disciplinary backgrounds shifting towards using R?

There are many answers to this question, but some of the most important are:

\begin{enumerate}
\def\labelenumi{\arabic{enumi}.}
\item
  R and RStudio are free.

  One of the biggest perks of working with R and RStudio is that both are available free of charge. Whereas other, proprietary statistics packages are often stuck in the dark ages of development (the 1990s, for example), and can be incredibly expensive to purchase, R is a free alternative that allows users of all experience levels to contribute to its development.
\item
  Analyses done using R are reproducible.

  As many scientific fields embrace the idea of reproducible analyses, proprietary point-and-click systems actually serve as a hindrance to this process. If you need to re-run your analysis using one of these systems, you'll need to carefully copy-and-paste your results into your text editor, potentially from beginning to end. As anyone who has done this sort of copy-and-pasting knows, this approach is both prone to errors and incredibly tedious.

  If, on the other hand, you use the workflows described in this book, your analyses will be reproducible, thus eliminating the copy-and-paste dance. And, as you can probably guess, it is much better to be able to update your code and data inputs and then re-run all of your analysis with the push of a button than to have to worry about manually moving your results from one program to another. Reproducibility also helps you as a programmer, since your most frequent collaborator is likely to be yourself a few months or years down the road. Instead of having to carefully write down all the steps you took to find the correct drop-down menu option, your entire code is stored, and immediately reusable.
\item
  Using R makes collaboration easier.

  This approach also helps with collaboration since, as you will see later, you can share a single R Markdown file containing all of your analysis, documentation, comments, and code with others. This reduces the time needed to work with others and reduces the likelihood of errors being made in following along with point-and-click analyses. The mantra here is to \textbf{Say No to Copy-And-Paste!} both for your sanity and for the sake of science.
\item
  Finding answers to questions is much simpler.

  If you have ever had an issue with software, you know how difficult it can be to find answers to your questions. ``How can I describe the process to someone else? Do I need to take screenshots? Do I really need to call IT and wait for hours for someone to respond?'' Because R is a programming language, it is much easier (after a bit of practice) to use Google or Stack Overflow to find answers to your questions. You'll be amazed at how many other users have encountered the same sorts of errors you will see when you begin.

  I frequently (almost daily) Google things like ``How do I make a side-by-side boxplot in R coloring by a third variable?''. You'll become better at working with R by reaching out to others for help and by answering questions that others have. In addition, Chapter \ref{errors} describes many common errors and explains how to fix them.
\item
  Struggling through programming helps you learn.

  We all know that learning isn't easy. Do you have trouble remembering how to follow a list of more than 10 steps or so? Do you find yourself going back over and over again because you can't remember what step comes next in the process? This is extremely common, especially if you haven't done the procedure in awhile. Learning by following a procedure is easy in the short-term, but can be extremely frustrating to remember in the long-term. If done well, programming promotes long-term thinking to short-term fixes.

  One unfortunate thing that we frequently take for granted is that our brain tricks us into picking the easy route. If you truly want to learn how to do something (like programming with R), you'll need to feel frustrated at times. Any time you learn something new, you've been frustrated. We tend to forget all the frustration and only think about where we currently are. Although R still frustrates me from time to time, I grow through practice and I look forward to the challenges. Garrett Grolemund encapsulated this phenomenon nicely in the Prologue of his book ``Hands-On Programming with R'' \citep{handson2014}:

  \begin{quote}
  As you learn to program, you are going to get frustrated. You are learning a new language, and it will take time to become fluent. But frustration is not just natural, it's actually a positive sign that you should watch for. Frustration is your brain's way of being lazy; it's trying to get you to quit and go do something easy or fun. If you want to get physically fitter, you need to push your body even though it complains. If you want to get better at programming, you'll need to push your brain. Recognize when you get frustrated and see it as a good thing: you're now stretching yourself. Push yourself a little further every day, and you'll soon be a confident programmer.
  \end{quote}
\end{enumerate}

\hypertarget{rstudiobasics}{%
\chapter{R and RStudio Basics}\label{rstudiobasics}}

\hypertarget{what-is-r}{%
\section{What is R?}\label{what-is-r}}

In Chapter \ref{whyR}, I discussed many of the reasons why you should begin doing your analyses (especially those of the data type) using R. If you skipped over that chapter in the hopes of just diving in to learning about R, I suggest you go back and read it over carefully. As you begin building fluency in working with R, it is especially important to review that introductory chapter from time to time.

\hypertarget{r-beginnings}{%
\subsection{R beginnings}\label{r-beginnings}}

R was developed by a group of statisticians who wanted an open-source alternative to the costly proprietary options that were (and still are) popular. Because it was created by statisticians (instead of computer scientists), R has some quirky aspects to it that take some time to get used to. We'll see that many packages have been developed to help with this, and these days, you don't need an advanced degree in statistics to work with R.

Getting back to the development of R\ldots{} R was created by \textbf{R}oss Ihaka and \textbf{R}obert Gentleman in New Zealand at the University of Auckland. It is a spin-off of the S programming language and was named partly after the first names of its developers (as you can see from the emphasis above). The beginning ideas for creating R came in 1992, and the first version of R was released in 1994. You can find much more about the background of R, its features, and its connections to the S language on \href{https://en.wikipedia.org/wiki/R_(programming_language)}{Wikipedia}.

\hypertarget{r-packages}{%
\subsection{R packages}\label{r-packages}}

I first learned to use R as a graduate student at Northern Arizona University from \href{http://www.stat.colostate.edu/~pturk/}{Dr.~Philip Turk} in 2007. At the time, I never thought that R could have exploded in users as we have seen since 2011. I never would have thought that students taking an introductory statistics course would be encouraged to learn to use R.

In 2007, R was still largely an esoteric and tricky language used by statisticians to do analyses. Getting used to the syntax for producing plots and working with data was especially tricky for those with little to no programming experience. So what has changed since 2007 about learning R?

I believe one of the biggest developments has been the creation of packages to make R easier to work with for newbies. Packages are add-ons created by users of R to increase the functionality of the base R installation. Packages created by Hadley Wickham and others recently have greatly expanded the capabilities of R, while also working to make beginning with R simpler. As of April 2017, \href{https://cran.r-project.org/web/packages/index.html}{more than 10,400 packages} were available on common R repositories.\footnote{You'll see how to download these packages via \texttt{install.packages("dplyr")} and load them into your current R working environment via \texttt{library("dplyr")}, for example, in Chapter \ref{rmdanal}.}

Another great development is the graphical user interface called RStudio and a package developed by RStudio, Inc.~called \texttt{rmarkdown}. We will discuss \texttt{rmarkdown} (also referred to as R Markdown) in a Chapter \ref{rmarkdown}, and will now focus on discussing RStudio.

\hypertarget{what-is-rstudio}{%
\section{What is RStudio?}\label{what-is-rstudio}}

RStudio is a powerful, free, open-source integrated development environment for R. Development on RStudio began in 2010, and the first beta was released in February 2011. It is available in two editions: RStudio Desktop and RStudio Server. This book will focus mostly on RStudio Server, but both versions are nearly identical to work with.

Instructions for downloading and installing R and RStudio on Windows and Mac machines are linked below. If you are using RStudio Server, your professor or members of your organization's IT department have done these steps for you. For RStudio Server, you log on using a web browser to an account on the cloud. There are many advantages to using the RStudio Server for beginning users, including sharing of R projects to help with feedback and error resolution. Installation of software can also cause its own headaches, which are eliminated by using the RStudio Server.

\textbf{Note for advanced users}: You can also install your very own RStudio Server for around \$5 per month on Digital Ocean. Instructions to do so can be found from Dean Attali \href{http://deanattali.com/2015/05/09/setup-rstudio-shiny-server-digital-ocean/}{here} and on the Digital Ocean site \href{https://www.digitalocean.com/community/tutorials/how-to-set-up-rstudio-on-an-ubuntu-cloud-server}{here}.

After you complete a few months of work with the RStudio Server, it is recommended that you download RStudio Desktop to your computer. The instructions to do so are below.

\hypertarget{installing-r-and-rstudio-desktop}{%
\subsection{Installing R and RStudio Desktop}\label{installing-r-and-rstudio-desktop}}

It is worth noting that you can't just install RStudio Desktop without first installing R, as RStudio needs to have R installed to run. Step-by-step guides to installing R and RStudio Desktop with screenshots are available

\begin{itemize}
\tightlist
\item
  \href{http://www.reed.edu/data-at-reed/software/R/r_studio.html}{here} for the Mac, and
\item
  \href{http://www.reed.edu/data-at-reed/software/R/r_studio_pc.html}{here} for a PC.
\end{itemize}

Unless you plan to create PDF documents (which requires a multiple gigabyte download of LaTeX), you can skip some of the later steps of the installation. It is recommended that you select \textbf{HTML} as the Default Output Format for R Markdown. You'll see more about this in Chapter \ref{rmarkdown}.

\hypertarget{working-in-rstudio-server}{%
\section{Working in RStudio Server}\label{working-in-rstudio-server}}

\hypertarget{logging-in-and-initial-screen}{%
\subsection{Logging in and initial screen}\label{logging-in-and-initial-screen}}

The RStudio Server provides a web-based interface to run analyses in R. This means that you will only need an internet connection and a web browser to run your analyses. Your professor or administrator will provide you with a link to the web location of your RStudio Server. After entering the link, you'll see a page that looks something like:

\begin{figure}

{\centering \includegraphics[width=\textwidth]{screenshots/server_login} 

}

\caption[Login page for RStudio Server]{Login page for RStudio Server}\label{fig:serverlogin}
\end{figure}

After logging in with your username and password, you should see a layout similar to what follows.

\begin{figure}

{\centering \includegraphics[width=\textwidth]{screenshots/initial_rstudio} 

}

\caption[Initial page for RStudio Server]{Initial page for RStudio Server}\label{fig:initialrstudio}
\end{figure}

For reference, a screenshot of RStudio Desktop looks similar:

\begin{figure}

{\centering \includegraphics[width=\textwidth]{screenshots/desktop_initial} 

}

\caption[Initial page for RStudio Desktop]{Initial page for RStudio Desktop}\label{fig:initialdesktop}
\end{figure}

This makes switching between the two RStudio set-ups painless. A discussion of each of the different RStudio panes and their corresponding tabs is in Chapter \ref{rmarkdown}. You'll find that a lot of what follows also applies to RStudio Desktop (except for the Shared Projects feature), but it is always recommended to create an RStudio project regardless of whether you are on the cloud or working locally.

\hypertarget{basic-workflow-with-rstudio}{%
\subsection{Basic Workflow with RStudio}\label{basic-workflow-with-rstudio}}

When starting a new R project, it is good practice to create a new RStudio project to go along with it. RStudio project files have the extension \texttt{.Rproj} and store metadata and information about the R environment you are working in. More information about RStudio projects is available from \href{https://support.rstudio.com/hc/en-us/articles/200526207-Using-Projects}{RStudio, Inc}.

If you are sharing homework or lab assignments with your instructor using RStudio Server, for example, it might make sense to create an RStudio project, share it with your instructor, and then create new folders for each lab. We will follow this example below.

The video below shows you how to create a new RStudio project called \texttt{initial} and add your first R Markdown file. Note that you also may see a description about what version of R is running on your initial login, as shown in the screencast below in the Console pane.

We have our \textbf{first\_rmarkdown.Rmd} file set up.

\hypertarget{sharing-projects-on-rstudio-server-pro}{%
\subsection{Sharing Projects on RStudio Server Pro}\label{sharing-projects-on-rstudio-server-pro}}

You will now see an example of how to share this project with another user. This will enable you and collaborators (other students, your instructor, etc.) to work on the \textbf{Rmd} file at the same time. This is similar to working collaboratively on a Google Doc.

RStudio Server is available in multiple formats, so you'll need to make sure you (or your IT administrator) have installed RStudio Server Pro to use the Shared Projects feature. You can find more information on this process from \href{https://support.rstudio.com/hc/en-us/articles/211659737-Sharing-Projects-in-RStudio-Server-Pro}{RStudio, Inc}. The video below illustrates the process of sharing the \textbf{initial.Rproj} project file we just created with another user on the same installation of RStudio Server.

Both myself and \texttt{bottk} can now work together on this project. We can type comments and code into \textbf{first\_rmarkdown.Rmd} or other files in the project and save files to the common folder where \textbf{initial.Rproj} resides.

In Chapter \ref{rmarkdown}, you'll see why it is recommended you work in R Markdown files and you'll also begin to see some examples of how R works with R Markdown.

\hypertarget{rstudio-layout}{%
\section{RStudio Layout}\label{rstudio-layout}}

Initially, you may be a little overwhelmed by all the different panes and tabs that are available in RStudio, but you will soon learn to appreciate this layout. We will begin with the top left pane and proceed clockwise. The layout of these panes can be customized, but it is recommended that beginning users keep the standard layout.

\hypertarget{code-editor-view-window}{%
\subsection{Code Editor / View Window}\label{code-editor-view-window}}

The pane in which you will likely spend a majority of your time is the pane on the top left. When you first login, this pane isn't there, but it appeared when we made the \textbf{first\_rmarkdown.Rmd}. This pane serves as a place to view the contents of files and objects in R. In the screencast below, you can see that we can change the text in this file and then save the file.

Note that, when you edit the file, the tab with the file name changes from black to red, and an asterisk appears after the file name, indicating that the latest changes have not been saved. You should get in the habit of saving files frequently. You can have multiple tabs open and view different files in this pane. In Chapter \ref{rmarkdown}, you'll also see that you can \texttt{View} datasets in this pane.

It may not be clear what these additions are really doing just yet. That's ok! When finished, you'll be pressing the \textbf{Knit HTML} button near the top of the pane to put all of your text, code, and its output together. We aren't quite there just yet though!

\hypertarget{environment-history}{%
\subsection{Environment / History}\label{environment-history}}

By default, the top right pane includes an \textbf{Environment} tab and a \textbf{History} tab. To get a sense of what these tabs provide, we'll need to also use the bottom left pane and the \textbf{Console} tab. I'll show you how to create multiple objects in R using the \textbf{Console}. Initially, you will see that the \textbf{Environment} tab tells us that the ``Environment is empty.'' This means there are no objects (e.g., data) yet. If you click on the \textbf{History} tab, you should also see a blank screen with a few icons. We won't go over all of these buttons here, but I encourage you to hover over them and click on them to get a sense for what they do. As I enter code into the \textbf{Console}, watch to see how the \textbf{Environment} and \textbf{History} tabs change.

You can think of the \textbf{Console} as a place to play around. It is your R sandbox. You can test your code to make sure it is working and then copy that text into your \textbf{Rmd} file once you are satisfied with it. We'll see more examples of this in the chapters to come.

Note that, by default, when you enter the name of an object into the console like I did with \texttt{sum\_1\_2} it displays the result. You've also been shown what is called the assignment operator denoted by \texttt{\textless{}-}. You can read this as putting the contents of the right-hand-side into an object named whatever appears on the left-hand-side. In the example, \texttt{num1} is the name of an object that stores the value 7.

A powerful feature of the R language has been introduced in the \texttt{sum\_1\_2\ \textless{}-\ sum(num1,\ num2)} line. \texttt{sum} is a function. Functions are denoted by their name and then a parenthesis, followed by one or more arguments separated by commas, and then a closing parenthesis. You'll see many examples of these going forward. Of course, we'll be using R for more than just a basic calculator shown here but this should give you an idea of what the \textbf{Environment} and \textbf{History} tabs store.

\hypertarget{console}{%
\subsection{Console}\label{console}}

You'll frequently use the \textbf{Console} as a way to check your work or experiment with how to solve a problem using R. Before RStudio, most users of R just had a window like the \textbf{Console} provides where they entered their commands and then looked at the results in different windows. We will see that the \textbf{Console} and the \textbf{Code Editor/View Window} will allow us to store all of our code in a file and then ``run'' that code through the \textbf{Console} to check that it works. The example video below shows how this is done.

\vspace*{0.2in}

\noindent\textbf{Rules for naming objects}\vspace*{0.1in}

It is good practice to get in the habit of naming variables corresponding to what they actually represent. If you are dividing two different sums of numbers, you might want to choose a name like \texttt{ratio\_of\_sums} to refer to that object. R has a few restrictions regarding what can be included in the name of R objects:

\begin{enumerate}
\def\labelenumi{\arabic{enumi}.}
\tightlist
\item
  Object names cannot begin with a number.
\item
  Object names cannot contain symbols used for mathematics or to denote other operations native to R. These symbols include \texttt{\$}, \texttt{@}, \texttt{!}, \texttt{\^{}}, \texttt{+}, \texttt{-}, \texttt{/}, and \texttt{*}.
\end{enumerate}

Another important property of R is that is case-sensitive. You'll see what this means in the video below that also includes examples of invalid names of objects. Note that we will continue to work inside the R chunk as we did in the last video. You'll see that these code chunks provide a nice way to keep track of our important analyses.

You may have noticed that R will do some checks and alert you to potential errors by placing a red X to the left of lines of code with common syntax errors. These checks won't catch all errors, but this can be helpful. Note also that \texttt{Name}, \texttt{name}, and \texttt{nAme} refer to three different values.

Another thing to notice is that you can store numbers into an object with a given name like we did earlier with \texttt{num1}. If we'd like to store a character string such as ``Chester'', we can provide the name of the object on the left-hand-side of \texttt{\textless{}-} and the string in quotation marks on the right-hand-side of \texttt{\textless{}-}. There are more complex object types that you will see in Chapter \ref{rmdanal}, but it is always important to think about the difference between a number and a character in R.

It's also a good idea to not call objects the names of functions that are built into R. You may want to call the addition of two numbers \texttt{sum} and R will allow this, but it is \textbf{HIGHLY} recommended that you create more descriptive names and not choose to name objects the same as common R functions. Something like \texttt{sum\_densities} is better and less likely to be the name of a function.

\hypertarget{the-help-function}{%
\subsection{\texorpdfstring{The \texttt{help} function (\texttt{?})}{The help function (?)}}\label{the-help-function}}

Of course, you won't know what all of the names of built-in functions are until you practice, but it is something to think about. If you are ever wondering if a function is built-in or is in a package you have included, you can use the \texttt{?} function in the \textbf{Console} to check. Some examples are shown below as a screencast.

\hypertarget{the-bottom-right-pane}{%
\subsection{The bottom right pane}\label{the-bottom-right-pane}}

The bottom right pane in RStudio contains the most tabs by default and is a useful place to view a variety of miscellaneous information about your RStudio project and its files.

\vspace*{0.2in}

\noindent\textbf{Files}\vspace*{0.1in}

The leftmost tab here shows the file and folder structure for the current working directory. In an RStudio project, this is the folder in which the project file was saved. This shows you where the files are stored, what they are called, and any folders that may exist in your project folder. This can be thought of as similar to going to \textbf{My Computer} on a PC or opening \textbf{Finder} on a Mac. Similarly, this tab lists the file and directory structure either on the cloud for RStudio Server or on your local machine for RStudio Desktop.

\vspace*{0.2in}

\noindent\textbf{Plots / Viewer}\vspace*{0.1in}

You'll see clearer examples of what the \textbf{Plots} and \textbf{Viewer} tabs provide in Chapter \ref{rmarkdown}. As you likely guessed \textbf{Plots} will show you the resulting graphs/figures that your R code has generated. The \textbf{Viewer} tab can show you the resulting HTML file created from an R Markdown \textbf{Knit}.

\vspace*{0.2in}

\noindent\textbf{Packages}\vspace*{0.1in}

The \textbf{Packages} tab lists all packages installed on your computer or cloud server. You can see which packages are loaded in the current working environment by looking to see if a check-mark exists next to the package name. Note that you may not have all of the packages loaded onto your machine that I do below in the video. That's OK. This is just an example of what you may expect.

You'll also notice a \textbf{Description} of the package as well as the \textbf{Version} number here. Packages are frequently updated and improved, so this is a way to check whether you have the most up-to-date version of a package. Remember, if you are using RStudio Server, this is likely taken care of for you. If you are using RStudio Desktop, you might find the \textbf{Install} and \textbf{Update} buttons useful for downloading new packages or updating currently installed ones.

\vspace*{0.2in}

\noindent\textbf{Help}\vspace*{0.1in}

We also saw an example of using the \textbf{Help} tab when we invoked the \texttt{?} function. This will show you documentation on R functions, datasets, and packages. When (not if) you come across code and you aren't sure what it does, it is often helpful to enter a question mark followed by the name, and see if the built-in documentation can help you out.

\hypertarget{rmarkdown}{%
\chapter{R Markdown}\label{rmarkdown}}

R Markdown provides an easy way to produce rich, fully-documented, reproducible analyses. It allows users to share a single file containing all of the comments, R code, and metadata needed to reproduce the analysis from beginning to end. R Markdown allows you to combine chunks of R code with Markdown text and produce a nicely formatted HTML, PDF, or Word file, without having to know any HTML or LaTeX code or fuss with getting the formatting just right in a Microsoft Word DOCX file.

One R Markdown file can generate a variety of different formats, and all of this is done using a single text file with a few bits of formatting. I think you'll be pleasantly surprised at how easy it is to write an R Markdown document once you get the hang of it.

\hypertarget{fixerrors}{%
\section{Fixing Errors in an R Markdown file}\label{fixerrors}}

Recall the R Markdown file (\textbf{first\_rmarkdown.Rmd}) that we created in Chapter \ref{rstudiobasics}. We know that we left some errors in the creation of variables there, and while it might seem strange to show you errors, it is good exposure for someone new to R to see a variety of the errors one might see initially. Let's see what happens when we click the \textbf{Knit HTML} button with these errors. Then, we will clean up the code and see what the resulting file looks like from the \textbf{Knit}.

When you first created this R Markdown file, a basic template was pre-populated with some code and text, to give you a sense of the kinds of things you can include in R Markdown files. We modified some of that code here. For example, I removed all of the lines in the code chunk named \texttt{cars} even though the errors did not occur in the declaration of the objects that had names stored in them. We see that an HTML file is produced in the \textbf{Viewer} pane, because \textbf{View in Pane} was selected.

As you look over the \textbf{Including Plots} text, you may be surprised to see that although there was no plot provided in the R Markdown file, the HTML file includes a scatter plot of temperature and pressure. This is because R Markdown evaluates the code stored in R chunks and then includes those results in the HTML (or PDF or DOCX, etc.) output.

You can also see that the text appears as commentary before and after the R code. You'll understand in a bit why the text ``Including Plots'' is so much larger than the other text.

\textbf{Important note}: Remember that all of the R code you want to run needs to be stored in a chunk (in the correct order) for your analysis to be reproducible AND for you not to receive errors when you \textbf{Knit}. It is easy to do a lot of work in the R \textbf{Console} and then forget to add that work into a chunk in your \textbf{Rmd} file. This is probably the number one error you will see when you first begin working in RStudio. An example of this error is shown below.

The \texttt{object\ not\ found} errors are the most frequently encountered errors, and along with misspellings and incomplete R code segments, represent the vast majority of issues with R. This is covered in greater detail in Chapter \ref{errors}.

\hypertarget{the-components-of-an-r-markdown-file}{%
\section{The Components of an R Markdown File}\label{the-components-of-an-r-markdown-file}}

\hypertarget{yaml}{%
\subsection{YAML}\label{yaml}}

The top part of the file is called the YAML header. YAML is a recursive acronym that stands for ``YAML Ain't Markup Language'' and is defined on its official website at \url{http://yaml.org} as:

\begin{quote}
YAML is a human friendly data serialization
standard for all programming languages.
\end{quote}

Essentially, the YAML header stores the metadata needed for the document. You can see an example of a YAML header from our \textbf{first\_rmarkdown.Rmd} file:

\begin{Shaded}
\begin{Highlighting}[]
\OtherTok{---}
\FunctionTok{title:}\AttributeTok{ }\StringTok{"First RMarkdown"}
\FunctionTok{author:}\AttributeTok{ }\StringTok{"Chester Ismay"}
\FunctionTok{output:}\AttributeTok{ html_document}
\OtherTok{---}
\end{Highlighting}
\end{Shaded}

There are many other fields that can be customized in the YAML header. The important thing to notice here are the three hyphens that begin and end the YAML header. Indentation also has meaning in YAML, so take care when aligning text.

\hypertarget{headers}{%
\subsection{Headers}\label{headers}}

As you can see above, you can create many different sized headers by simply adding one or more \texttt{\#} in front of the text you'd like to denote the header.

\hypertarget{emphasis}{%
\subsection{Emphasis}\label{emphasis}}

Whenever you see a hash-tag in the text of your R Markdown document, you now know that this will correspond to bolded, larger text\footnote{Unless you want to have a fourth, fifth, or sixth level header, but these are not common.} that denotes the start of a section of your document.

This is one of the nice features of R Markdown. You can simply look at the plain text and know what it will produce in the knitted document. We can also add different styles of emphasis to words, phrases, or sentences by surrounding them in matching symbols. Below are some examples.

You are beginning to see how easy it is to customize your output. We'll next discuss ways to add links to URLs, create ordered and unordered lists, and use other frequently used Markdown features.

\hypertarget{links}{%
\subsection{Links}\label{links}}

To add a link to a URL, you simply enclose the text you'd like displayed in the resulting HTML file inside \texttt{{[}\ {]}} and then the link itself inside \texttt{(\ )} right next to each other with no space in between.

\hypertarget{lists}{%
\subsection{Lists}\label{lists}}

The screencast below shows the process of creating both ordered and unordered lists.

Note that only numbers are needed as we saw by numbering ``Warm up food'' with a ``1.'' We can also combine unordered and ordered lists by indenting the text two spaces.

In many of the examples that follow, you will see the actual text you'd type into your R Markdown document highlighted with a gray background and also the results of that text immediately below it.

\begin{Shaded}
\begin{Highlighting}[]
\NormalTok{1. }\FloatTok{Wake up}
\FloatTok{  - Get out of bed}
\FloatTok{1. Warm up food}
\FloatTok{  - Open kitchen door}
\FloatTok{  - Get plate out of cupboard}
\FloatTok{2. Make coffee}
\FloatTok{  i. Warm up water}
\FloatTok{  ii. Grind beans}
\FloatTok{3. Make breakfast}

\NormalTok{  We can have a paragraph (or two) here describing how we could go about making}
\NormalTok{  breakfast. If we indent the paragraph a few spaces and create a newline, it}
\NormalTok{  will indent below the item.}
\end{Highlighting}
\end{Shaded}

\begin{enumerate}
\def\labelenumi{\arabic{enumi}.}
\item
  Wake up

  \begin{itemize}
  \tightlist
  \item
    Get out of bed
  \end{itemize}
\item
  Warm up food

  \begin{itemize}
  \tightlist
  \item
    Open kitchen door
  \item
    Get plate out of cupboard
  \end{itemize}
\item
  Make coffee

  \begin{enumerate}
  \def\labelenumii{\roman{enumii}.}
  \tightlist
  \item
    Warm up water
  \item
    Grind beans
  \end{enumerate}
\item
  Make breakfast

  We can have a paragraph (or two) here describing how we could go about making breakfast. If we indent the paragraph a few spaces it will indent below the item.
\end{enumerate}

\hypertarget{miscellaneous-markdown}{%
\subsection{Miscellaneous Markdown}\label{miscellaneous-markdown}}

\textbf{Line breaks / white spacing}

Line breaks in combination with white space are incredibly important in Markdown, as they frequently denote the start of a new paragraph.

\begin{Shaded}
\begin{Highlighting}[]
\NormalTok{Here is an example of text with only a line break.}
\NormalTok{You may expect this line to appear in a new paragraph but it doesn't.}
\end{Highlighting}
\end{Shaded}

Here is an example of text with only a line break.
You may expect this line to appear in a new paragraph but it doesn't.

In order to start a new paragraph, you need to add white space between the two paragraphs:

\begin{Shaded}
\begin{Highlighting}[]
\NormalTok{Here is an example of text with a line break and white space.}

\NormalTok{You may expect this line to appear in a new paragraph and it does.}
\end{Highlighting}
\end{Shaded}

Here is an example of text with a line break and white space.

You may expect this line to appear in a new paragraph and it does.

\textbf{Horizontal rules}

Another useful way to divide up different parts of your document is by including horizontal lines, which can be added by placing three asterisks (or three hyphens) next to each other:

\begin{Shaded}
\begin{Highlighting}[]
\NormalTok{***}
\end{Highlighting}
\end{Shaded}

\begin{center}\rule{0.5\linewidth}{\linethickness}\end{center}

\begin{Shaded}
\begin{Highlighting}[]
\NormalTok{---}
\end{Highlighting}
\end{Shaded}

\begin{center}\rule{0.5\linewidth}{\linethickness}\end{center}

\textbf{Blockquotes}

If you'd like to quote someone or produce an indented text block, you can do so by adding a \texttt{\textgreater{}} before the passage:

\begin{Shaded}
\begin{Highlighting}[]
\NormalTok{>}\DataTypeTok{ Reproducible research is the idea that data analyses, and more generally,}
\DataTypeTok{scientific claims, are published with their data and software code so that}
\DataTypeTok{others may verify the findings and build upon them. - Roger Peng}
\end{Highlighting}
\end{Shaded}

\begin{quote}
Reproducible research is the idea that data analyses, and more generally, scientific claims, are published with their data and software code so that others may verify the findings and build upon them. - Roger Peng
\end{quote}

\textbf{Commenting Text}

There are times where you might want to comment out text inside an R Markdown document. Say you wrote something that you don't really want in the resulting knitted document, but you aren't quite sure if you should delete it completely. To create a comment, enclose the block of text in \texttt{\textless{}!-\/-\ -\/-\textgreater{}} as seen below:

\begin{Shaded}
\begin{Highlighting}[]
\NormalTok{<!--}
\CommentTok{I'd like to save this text for later and don't want to delete it yet.}
\CommentTok{-->}
\end{Highlighting}
\end{Shaded}

You won't see the commented out results here in the book.

\vspace*{0.2in}

\noindent\textbf{Equations}\vspace*{0.1in}

If you'd like nice mathematical formulas in your document, you can add them between two dollar signs:

\begin{Shaded}
\begin{Highlighting}[]
\NormalTok{$y = mx + b$}
\end{Highlighting}
\end{Shaded}

\(y = mx + b\)

\hypertarget{r-chunks}{%
\subsection{R Chunks}\label{r-chunks}}

Now that you have the basics down, we can get to what I believe is the best part about R Markdown: the ability to include R code directly in the document which is compiled in the resulting output. You've seen several examples of R chunks in the R Markdown file already. These code blocks all share several properties in common which you should know:

\begin{itemize}
\tightlist
\item
  Code blocks always begin and end with three backticks \texttt{\textasciigrave{}\textasciigrave{}\textasciigrave{}}.
\item
  After the initial three backticks, the first line of an R chunk begins with \texttt{\{r}, optionally includes a name and/or other chunk options, and then ends with a \texttt{\}}.
\item
  The lines enclosed between the beginning and closing three backticks is valid R code that you could execute in the console.
\end{itemize}

Note that including spaces in front of these backticks will produce an error.

In our \textbf{first\_rmarkdown.Rmd} file, let's explore an example of recognizing and creating our own R chunks:

This example introduces you to two different ways to create a vector of values. You'll see further discussion of this in Chapter \ref{rmdanal}. You can see that the code was automatically executed when we pressed the \textbf{Knit HTML} button, and its output was included in the knitted file.

\textbf{Important note}: Any other R chunks after this one will have access to the three variables created here: \texttt{count20}, \texttt{count100\_by\_5}, and \texttt{prod}. Any chunks before the \texttt{mult\_vectors} named chunk \textbf{\emph{WILL NOT}} have access to these variables. You can read the document like a book, so it is important to add objects and work with them in the appropriate order. You'll receive errors from R if you don't.

\hypertarget{inline-r-code}{%
\subsection{Inline R code}\label{inline-r-code}}

We've seen that we can add R code and have that run in an R chunk of code enclosed by three backticks. However, what if we wanted to include the results of a simple calculation directly in the text of our document? R Markdown can do that as well:

Another crafty approach is to have the text produced in our document automatically update based on the results of R code. To see an example of this, we will select a number at random from our \texttt{count20} vector. If the number is greater than or equal to 10, we will say so. If it isn't, we will report that.

You also see that R gave an error when I didn't include the third argument to \texttt{ifelse} in quotation marks. However, when I fixed the code and pressed \textbf{Knit HTML} again, the error went away.

\hypertarget{code-highlighting}{%
\subsection{Code Highlighting}\label{code-highlighting}}

As you saw in the previous example, it is a good habit to highlight the names of your R objects to differentiate them from regular text. This can be done by enclosing the word in a single backtick such as what we did with \texttt{one\_value}.

\hypertarget{r-markdown-chunk-options}{%
\section{R Markdown Chunk Options}\label{r-markdown-chunk-options}}

You can set many options on a chunk by chunk basis. The most common R chunk options are \texttt{echo}, \texttt{eval}, and \texttt{include}. By default, all three of these options are set to \texttt{TRUE}.

\begin{itemize}
\item
  \texttt{echo} dictates whether the code that produces the result should be printed before the corresponding R output
\item
  \texttt{eval} specifies whether the code should be evaluated or just displayed without its output
\item
  \texttt{include} specifies whether the code AND its output should be included in the resulting knitted document. If it is set to \texttt{FALSE} the code is run, but neither the code or its output are included in the resulting document.
\end{itemize}

Because we specified that \texttt{eval=FALSE} and that chunk was where we declared the \texttt{one\_value} variable, we now obtain an error. You can include multiple chunk options by separating each option with a comma.

\hypertarget{general-guidelines-for-writing-r-markdown-files}{%
\section{General Guidelines for Writing R Markdown Files}\label{general-guidelines-for-writing-r-markdown-files}}

White space is your friend. You should always include a blank white space between R chunks and your Markdown text. It makes your document much more readable and can reduce some potential errors. Also, leave a line of white space between header text and your paragraphs.

Commentary is always good. Explain yourself and your ideas whenever you can. Remember that your most frequent collaborator is likely yourself a few months down the road. Be nice to future you and explain what you are doing so that you can remember!

Remember that the Console and R Markdown environments (when Knitting) don't interact with each other. This forces you to include only the code in your R chunks that produces exactly the results you want to share with others. Don't inflate your document with extra output. Be concise and clear in exactly what you are doing.

The chunk options can really beautify your documents and customize them exactly to what you'd like the reader of your documents to see. You can find more information on all of the available R chunk options \href{http://yihui.name/knitr/options/}{here}.

\hypertarget{help---cheatsheets}{%
\section{Help -\textgreater{} Cheatsheets}\label{help---cheatsheets}}

RStudio provides really nice cheatsheets that can act as great references to many of the common tasks you will do inside of RStudio. You can get nice PDF versions of the files by going to \textbf{Help -\textgreater{} Cheatsheets} inside RStudio.

\vspace{0.1in}\begin{center}\footnotesize{\url{https://raw.githubusercontent.com/ismayc/rbasics-book/gh-pages/screenshots/cheatsheets.png}}\end{center}\vspace{0.1in}

\hypertarget{spell-check}{%
\section{Spell-check}\label{spell-check}}

Near the top of your R Markdown editor window sits one of the more useful tools for writing documents: the spell-check button. It is the green check-mark with ``ABC'' above it:

\begin{center}\includegraphics[width=0.3\textwidth]{images/spellcheck} \end{center}

Before you submit a document or share it with someone else, you should spell check your document. You may need to add some R commands to the dictionary or ignore them since those may not be recognized as words, but it is easy to misspell words as we type and this feature can really help.

\hypertarget{rmdanal}{%
\chapter{Intro to R using R Markdown}\label{rmdanal}}

In this chapter, you'll see many of the ways that R stores objects and more details on how you can use functions to solve problems in R. In doing so, you will be working with a common dataset derived from something that you likely have encountered before: the periodic table of elements from chemistry.

\hypertarget{a-beginning-directoryfile-workflow}{%
\section{A beginning directory/file workflow}\label{a-beginning-directoryfile-workflow}}

\begin{quote}
``File organization and naming are powerful weapons against chaos.'' - Jenny Bryan
\end{quote}

Something that is not frequently discussed when working with files and programming languages like R is the importance of naming your files something relevantly unique and organizing your files in folders. You may be tempted to call a file \texttt{analysis.Rmd} but what happens when you need to change your analysis months from now and you've named many files for many different projects \texttt{analysis.Rmd} in many different folders. It's much better to give your future self a break and commit to concise naming strategies.

You can choose a variety of ways to name files. These guidelines are what I try to follow:

\begin{enumerate}
\def\labelenumi{\arabic{enumi}.}
\tightlist
\item
  Group similar style files into the same folder whenever possible.
\end{enumerate}

Organize files by type: Try not to have one folder that contains all of the
different types of files you are working with. It is much easier to find what you
are looking for if you put all of your \texttt{data} files in a \texttt{data} folder, all of
your \texttt{figure} files in a \texttt{figure} folder, and so on.

This rule can be broken if you only have one or two files of each type. In that
case, too many folders can make it harder to find what you're looking for, in which
case, searching may be faster than digging through a complex hierarchy of directories.

\begin{enumerate}
\def\labelenumi{\arabic{enumi}.}
\setcounter{enumi}{1}
\tightlist
\item
  Name files consistently so that the contents of the file can be easily
  identified from the name of your file, but be concise.
\end{enumerate}

Seeing \texttt{test1.Rmd} and \texttt{test2.Rmd} doesn't tell us much about what is actually
in the files. It's OK to create a temporary file or two if you don't think you
will be using it going forward, but you should be in the habit of reviewing your
work at the end of your session and renaming useful files appropriately.
Something like \texttt{model\_fit\_sodium.Rmd} is so much better in the long-run.
Remember to think about your future self whenever you can, especially with
programming. Be nice to yourself so that future self really appreciates past self.

\begin{enumerate}
\def\labelenumi{\arabic{enumi}.}
\setcounter{enumi}{2}
\tightlist
\item
  Use an underscore to separate elements of the file name.
\end{enumerate}

The underscore \texttt{\_} produces a visually appealing way for us to realize there is
a separation between content. You may be tempted to just use a space, but spaces
cause problems in R and in other programming languages. Some folks prefer to just
change the case of the first letter of the word to bring attention. Here are a few
examples of file names. You can be the judge as to what is most appealing to you:

\texttt{barplot\_weight\_height.Rmd} vs \texttt{barplotWeightHeight.Rmd} vs \texttt{barplotweightheight.Rmd}

Whatever you choose for style, be consistent and think about other users as you
name your files. If you were given a smorgasbord of files that is a mess to
deal with and hard to understand, you wouldn't like it, right? Don't be that
person to someone else (or yourself)!

\hypertarget{using-r-with-periodic-table-dataset}{%
\section{Using R with periodic table dataset}\label{using-r-with-periodic-table-dataset}}

We now dive in to the basics of working with a dataset in R. We will explore the ways R stores data in \textbf{objects}, how to access specific elements in those objects, and how to use \textbf{functions}, which are one of the most useful pieces of R, to help with organization and clean code. After completing this introduction, you will be prepared to dive in to statistical analyses or data tasks of your own.

It is worth noting that many of the functions described here such as \texttt{table} and concepts like subsetting, indexing, and creating/modifying new variables can also be done using the great packages that Hadley Wickham has developed, in particular the \texttt{dplyr} package. Nevertheless, it is still important to get a sense for how R stores objects and how to interact with objects in the ``old-school'' way. You'll still find some times where doing it this way actually works just as well as the newer modern ways\ldots{}but those times are becoming less frequent all the time.

\hypertarget{loading-data-from-a-file}{%
\subsection{Loading data from a file}\label{loading-data-from-a-file}}

One of the most common ways you'll want to work with data is by importing it from a file. A common file format that works nicely with R is the CSV (comma-separated values) file. The following R commands download the CSV file from the internet into the \texttt{periodic-table-data.csv} file on your computer, reads in the CSV stored, and then gives the name \texttt{periodic\_table} to the data frame that stores these values in R:

\begin{Shaded}
\begin{Highlighting}[]
\KeywordTok{download.file}\NormalTok{(}\DataTypeTok{url =} \StringTok{"http://ismayc.github.io/periodic-table-data.csv"}\NormalTok{,}
 \DataTypeTok{destfile =} \StringTok{"periodic-table-data.csv"}\NormalTok{)}
\end{Highlighting}
\end{Shaded}

\begin{Shaded}
\begin{Highlighting}[]
\NormalTok{periodic_table <-}\StringTok{ }\KeywordTok{read.csv}\NormalTok{(}\StringTok{"periodic-table-data.csv"}\NormalTok{,}
              \DataTypeTok{stringsAsFactors =} \OtherTok{FALSE}\NormalTok{)}
\end{Highlighting}
\end{Shaded}

We will be discussing both \textbf{strings} and \textbf{factors} in the data structures section (Section \ref{data-structures}) and why this additional parameter \texttt{stringsAsFactors} set to \texttt{FALSE} is recommended.

It's good practice to check out the data after you have loaded it in:

\begin{Shaded}
\begin{Highlighting}[]
\KeywordTok{View}\NormalTok{(periodic_table)}
\end{Highlighting}
\end{Shaded}

The video below walks through downloading the CSV file and loading it into the data frame object named \texttt{periodic\_table}. It also shows another way to view data frames that is built into RStudio without having to run the \texttt{View} function. Note that a new R Markdown file is created here as well called \texttt{chemistry\_example.Rmd}. We are writing the commands directly into R chunks here, but you may find it easier to play around in your R Console sandbox first.

I encourage you to take the R chunks that follow in this chapter, type/copy them into an R Markdown file that has loaded the \texttt{periodic\_table} data set, and see that your resulting output matches the output that I have presented for you here. This book was written in R Markdown after all!

\hypertarget{data-structures}{%
\section{Data structures}\label{data-structures}}

\hypertarget{data-frames}{%
\subsection{Data frames}\label{data-frames}}

Data frames are by far the most common type of object you will work with in R. This makes sense since R is a statistical computing language at its core, so handling spreadsheet-like data is something it should be good at. The \texttt{periodic\_table} data set is stored as a data frame. As you can see, this data set includes many different types of variables. We can get a sense of the types and some of the values of the variables by using the \texttt{str} function:

\begin{Shaded}
\begin{Highlighting}[]
\KeywordTok{str}\NormalTok{(periodic_table)}
\end{Highlighting}
\end{Shaded}

Each of the names of the variables/columns in the data frame are listed immediately after the \texttt{\$}. Then on each row of the \texttt{str} function call after the \texttt{:} we see what type of variable it is. The \texttt{periodic\_table} data set has four data types: \texttt{int}, \texttt{chr}, \texttt{num}, and \texttt{logi}:

\begin{itemize}
\tightlist
\item
  \texttt{int} corresponds to integer values
\item
  \texttt{chr} corresponds to character string values
\item
  \texttt{num} corresponds to numeric (not necessarily integer) values
\item
  \texttt{logi} corresponds to logical values (\texttt{TRUE} or \texttt{FALSE})
\end{itemize}

\hypertarget{vectors}{%
\subsection{Vectors}\label{vectors}}

Data frames can be thought of as many vectors of the same length put together into a single object. In the \texttt{periodic\_table} data frame, each row corresponds to a chemical element and each column corresponds to a different measurement or characteristic of that element. There are many different ways to create a vector that stands on its own outside of a data frame. \newline\vspace*{0.1in}

\vspace*{0.2in}

\noindent\textbf{Using the c function}\vspace*{0.1in}

If you would like to list out many entries and put them into a vector object, you can do so via the \texttt{c} function. If you enter \texttt{?c} in the R Console, you can gain information about it. The ``c'' stands for combine or concatenate.

Suppose we wanted a way to store four names:

\begin{Shaded}
\begin{Highlighting}[]
\NormalTok{friend_names <-}\StringTok{ }\KeywordTok{c}\NormalTok{(}\StringTok{"Bertha"}\NormalTok{, }\StringTok{"Herbert"}\NormalTok{, }\StringTok{"Alice"}\NormalTok{, }\StringTok{"Nathaniel"}\NormalTok{)}
\NormalTok{friend_names}
\end{Highlighting}
\end{Shaded}

\begin{verbatim}
## [1] "Bertha"    "Herbert"   "Alice"     "Nathaniel"
\end{verbatim}

You can see when \texttt{friend\_names} is outputted that there are four entries to it. This is vector is known as a \textbf{strings} vector since it contains character strings. You can check to see what type an object is by using the \texttt{class} function:

\begin{Shaded}
\begin{Highlighting}[]
\KeywordTok{class}\NormalTok{(friend_names)}
\end{Highlighting}
\end{Shaded}

\begin{verbatim}
## [1] "character"
\end{verbatim}

Next suppose we wanted to put the ages of our friends in another vector. We can again use the \texttt{c} function:

\begin{Shaded}
\begin{Highlighting}[]
\NormalTok{friend_ages <-}\StringTok{ }\KeywordTok{c}\NormalTok{(25L, 37L, 22L, 30L)}
\NormalTok{friend_ages}
\end{Highlighting}
\end{Shaded}

\begin{verbatim}
## [1] 25 37 22 30
\end{verbatim}

\begin{Shaded}
\begin{Highlighting}[]
\KeywordTok{class}\NormalTok{(friend_ages)}
\end{Highlighting}
\end{Shaded}

\begin{verbatim}
## [1] "integer"
\end{verbatim}

Note the use of the \texttt{L} value here. This tells R that the numbers entered have no decimal components. If we didn't designate the \texttt{L} we can see that the values are read in as \texttt{"numeric"} by default:

\begin{Shaded}
\begin{Highlighting}[]
\NormalTok{ages_numeric <-}\StringTok{ }\KeywordTok{c}\NormalTok{(}\DecValTok{25}\NormalTok{, }\DecValTok{37}\NormalTok{, }\DecValTok{22}\NormalTok{, }\DecValTok{30}\NormalTok{)}
\KeywordTok{class}\NormalTok{(ages_numeric)}
\end{Highlighting}
\end{Shaded}

\begin{verbatim}
## [1] "numeric"
\end{verbatim}

From a user's perspective, there is not a huge difference in how these values are stored, but it is still a good habit to specify what class your variables are whenever possible to help with collaboration and documentation. \newline

\vspace*{0.2in}

\noindent\textbf{Using the seq function}\vspace*{0.1in}

The most likely way you will enter character values into a vector is via the \texttt{c} function. Numeric values can be entered in a couple different ways. One is using the \texttt{c} function, as we saw above. Because numbers have a natural order, we can also specify a sequence of numbers with a starting value, an ending value, and the amount by which to increment each step in the sequence:

\begin{Shaded}
\begin{Highlighting}[]
\NormalTok{sequence_by_}\DecValTok{2}\NormalTok{ <-}\StringTok{ }\KeywordTok{seq}\NormalTok{(}\DataTypeTok{from =}\NormalTok{ 0L, }\DataTypeTok{to =}\NormalTok{ 100L, }\DataTypeTok{by =}\NormalTok{ 2L)}
\NormalTok{sequence_by_}\DecValTok{2}
\end{Highlighting}
\end{Shaded}

\begin{verbatim}
##  [1]   0   2   4   6   8  10  12  14  16  18  20  22  24  26  28  30  32
## [18]  34  36  38  40  42  44  46  48  50  52  54  56  58  60  62  64  66
## [35]  68  70  72  74  76  78  80  82  84  86  88  90  92  94  96  98 100
\end{verbatim}

\begin{Shaded}
\begin{Highlighting}[]
\KeywordTok{class}\NormalTok{(sequence_by_}\DecValTok{2}\NormalTok{)}
\end{Highlighting}
\end{Shaded}

\begin{verbatim}
## [1] "integer"
\end{verbatim}

You should now have a better sense of what the numbers in the \texttt{{[}\ {]}} before the output refer to. This helps you keep track of where you are in the printing of the output. So the first element denoted by \texttt{{[}1{]}} is 0, the 18\textsuperscript{th} entry (\texttt{{[}18{]}}) is 34, and the 35\textsuperscript{th} entry (\texttt{{[}35{]}}) is 68. This will serve as a nice introduction into indexing and subsetting in Section \ref{index-sub}.

We can also set the sequence to go by a negative number or a decimal value. We will do both in the next example.

\begin{Shaded}
\begin{Highlighting}[]
\NormalTok{dec_frac_seq <-}\StringTok{ }\KeywordTok{seq}\NormalTok{(}\DataTypeTok{from =} \DecValTok{10}\NormalTok{, }\DataTypeTok{to =} \DecValTok{3}\NormalTok{, }\DataTypeTok{by =} \FloatTok{-0.2}\NormalTok{)}
\NormalTok{dec_frac_seq}
\end{Highlighting}
\end{Shaded}

\begin{verbatim}
##  [1] 10.0  9.8  9.6  9.4  9.2  9.0  8.8  8.6  8.4  8.2  8.0  7.8  7.6
## [14]  7.4  7.2  7.0  6.8  6.6  6.4  6.2  6.0  5.8  5.6  5.4  5.2  5.0
## [27]  4.8  4.6  4.4  4.2  4.0  3.8  3.6  3.4  3.2  3.0
\end{verbatim}

\begin{Shaded}
\begin{Highlighting}[]
\KeywordTok{class}\NormalTok{(dec_frac_seq)}
\end{Highlighting}
\end{Shaded}

\begin{verbatim}
## [1] "numeric"
\end{verbatim}

\vspace*{0.2in}

\noindent\textbf{Using the : operator}\vspace*{0.1in}

A short-cut version of the \texttt{seq} version can be achieved using the \texttt{:} operator. If we are increasing values by 1 (or -1), we can use the \texttt{:} operator to build our vector:

\begin{Shaded}
\begin{Highlighting}[]
\NormalTok{inc_seq <-}\StringTok{ }\DecValTok{98}\OperatorTok{:}\DecValTok{112}
\NormalTok{inc_seq}
\end{Highlighting}
\end{Shaded}

\begin{verbatim}
##  [1]  98  99 100 101 102 103 104 105 106 107 108 109 110 111 112
\end{verbatim}

\begin{Shaded}
\begin{Highlighting}[]
\NormalTok{dec_seq <-}\StringTok{ }\DecValTok{5}\OperatorTok{:-}\DecValTok{5}
\NormalTok{dec_seq}
\end{Highlighting}
\end{Shaded}

\begin{verbatim}
##  [1]  5  4  3  2  1  0 -1 -2 -3 -4 -5
\end{verbatim}

\vspace*{0.2in}

\noindent\textbf{Combining vectors into data frames}\vspace*{0.1in}

If you aren't reading in data from a file and you have some vectors of information you'd like combined into a single data frame, you can use the \texttt{data.frame} function:

\begin{Shaded}
\begin{Highlighting}[]
\NormalTok{friends <-}\StringTok{ }\KeywordTok{data.frame}\NormalTok{(}\DataTypeTok{names =}\NormalTok{ friend_names,}
           \DataTypeTok{ages =}\NormalTok{ friend_ages,}
           \DataTypeTok{stringsAsFactors =} \OtherTok{FALSE}\NormalTok{)}
\NormalTok{friends}
\end{Highlighting}
\end{Shaded}

\begin{verbatim}
##       names ages
## 1    Bertha   25
## 2   Herbert   37
## 3     Alice   22
## 4 Nathaniel   30
\end{verbatim}

Here we have created a \texttt{names} variable in the \texttt{friends} data frame that corresponds to the values in the \texttt{friend\_names} vector and similarly an \texttt{ages} variable in \texttt{friends} that corresponds to the values in \texttt{friend\_ages}.

\hypertarget{factors}{%
\subsection{Factors}\label{factors}}

If we have a strings vector/variable that has some sort of natural ordering to it, it frequently makes sense to convert that vector/variable into a \textbf{factor}. The factor will convert the strings to integers to keep track of which order you'd prefer, and also keep track of the original string values as well.

Looking over our \texttt{periodic\_table} data frame again via \texttt{View(periodic\_table)}, you can see some good candidates for shifting from \texttt{chr} to \texttt{Factor}. If you remember your chemistry, you'll know that the natural ordering of the \texttt{block} variable is \texttt{"s"}, \texttt{"p"}, \texttt{"d"}, and \texttt{"f"}.

By default, R will organize character strings in alphabetical order. To see this, we'll introduce two new features: the \texttt{table} function and the \texttt{\$} operator.

\begin{Shaded}
\begin{Highlighting}[]
\KeywordTok{table}\NormalTok{(periodic_table}\OperatorTok{$}\NormalTok{block)}
\end{Highlighting}
\end{Shaded}

\begin{verbatim}
## 
##  d  f  p  s 
## 40 28 36 14
\end{verbatim}

The table function provides a count of the number of elements that appear in each block. But as you can see, the ordering is off. You may remember the \texttt{\$} displayed before variable names in the \texttt{str} function. That isn't a coincidence. To access specific variables inside a data frame, we can do so by entering the name of the data frame followed by \texttt{\$} and the name of the variable. (Note that spaces in variable names will not work. You'll likely learn that the hard way, as I have.)

To convert \texttt{block} into a factor, we use the aptly named \texttt{factor} function. Note that this is ``converting'' by assigning the result of \texttt{factor} back to \texttt{block} in \texttt{periodic\_table}.

\begin{Shaded}
\begin{Highlighting}[]
\NormalTok{periodic_table}\OperatorTok{$}\NormalTok{block <-}\StringTok{ }\KeywordTok{factor}\NormalTok{(periodic_table}\OperatorTok{$}\NormalTok{block,}
                \DataTypeTok{levels =} \KeywordTok{c}\NormalTok{(}\StringTok{"s"}\NormalTok{, }\StringTok{"p"}\NormalTok{, }\StringTok{"d"}\NormalTok{, }\StringTok{"f"}\NormalTok{))}
\end{Highlighting}
\end{Shaded}

\begin{Shaded}
\begin{Highlighting}[]
\KeywordTok{table}\NormalTok{(periodic_table}\OperatorTok{$}\NormalTok{block)}
\end{Highlighting}
\end{Shaded}

\begin{verbatim}
## 
##  s  p  d  f 
## 14 36 40 28
\end{verbatim}

You'll find that this is an easy way to organize your data whenever you'd like to summarize it or to plot it, but we'll save that discussion for a different time and a different book.

\hypertarget{vectorized-operations}{%
\section{Vectorized operations}\label{vectorized-operations}}

R can work extremely quickly when provided with a vector or a collection of vectors like a data frame. Instead of iterating through each element to perform an operation that we might need to do in other programming languages, we can do something like this:

\begin{Shaded}
\begin{Highlighting}[]
\NormalTok{five_years_older <-}\StringTok{ }\NormalTok{ages_numeric }\OperatorTok{+}\StringTok{ }\NormalTok{5L}
\NormalTok{five_years_older}
\end{Highlighting}
\end{Shaded}

\begin{verbatim}
## [1] 30 42 27 35
\end{verbatim}

Just like that, every age is five more than where we started. This extends to adding two vectors together\footnote{Vectors of the same size, of course\ldots{}well, actually R has a way of dealing with vectors of different sizes and not giving errors, but let's ignore that for now.}.

\hypertarget{index-sub}{%
\section{Indexing and subsetting}\label{index-sub}}

So we have a big data frame of information about the periodic table, but what if we wanted to extract smaller pieces of the data frame? You already saw that to focus on any specific variable we can use the \texttt{\$} operator.

\hypertarget{using-with-a-vectorvariable}{%
\subsection{\texorpdfstring{Using \texttt{{[}\ {]}} with a vector/variable}{Using {[} {]} with a vector/variable}}\label{using-with-a-vectorvariable}}

Recall the use of \texttt{{[}\ {]}} when a vector was printed, to help us better understand where we were in printing a large vector. We can use this same tool to select the tenth to the twentieth elements of the \texttt{periodic\_table\$name} variable:

\begin{Shaded}
\begin{Highlighting}[]
\NormalTok{periodic_table}\OperatorTok{$}\NormalTok{name[}\DecValTok{10}\OperatorTok{:}\DecValTok{20}\NormalTok{]}
\end{Highlighting}
\end{Shaded}

\begin{verbatim}
##  [1] "Neon"       "Sodium"     "Magnesium"  "Aluminium"  "Silicon"   
##  [6] "Phosphorus" "Sulfur"     "Chlorine"   "Argon"      "Potassium" 
## [11] "Calcium"
\end{verbatim}

Similarly, if we only want to select a few elements from our \texttt{friend\_names} vector, we can specify the entries directly:

\begin{Shaded}
\begin{Highlighting}[]
\NormalTok{friend_names[}\KeywordTok{c}\NormalTok{(}\DecValTok{1}\NormalTok{, }\DecValTok{3}\NormalTok{)]}
\end{Highlighting}
\end{Shaded}

\begin{verbatim}
## [1] "Bertha" "Alice"
\end{verbatim}

We can also use \texttt{-} to select everything but the elements listed after it:

\begin{Shaded}
\begin{Highlighting}[]
\NormalTok{friend_names[}\OperatorTok{-}\KeywordTok{c}\NormalTok{(}\DecValTok{2}\NormalTok{, }\DecValTok{4}\NormalTok{)]}
\end{Highlighting}
\end{Shaded}

\begin{verbatim}
## [1] "Bertha" "Alice"
\end{verbatim}

\hypertarget{using-with-a-data-frame}{%
\subsection{\texorpdfstring{Using \texttt{{[}\ ,\ {]}} with a data frame}{Using {[} , {]} with a data frame}}\label{using-with-a-data-frame}}

You have now seen how to select specific elements of a vector or a variable, but what if we want a subset of the values in the full data frame across both rows (observations) and columns (variables). We can use \texttt{{[}\ ,\ {]}} where the spot before the comma corresponds to rows and the spot after the comma corresponds to columns. Let's select rows 40 to 50 and columns 1, 2, and 4 from \texttt{periodic\_table}:

\begin{Shaded}
\begin{Highlighting}[]
\NormalTok{periodic_table[}\DecValTok{41}\OperatorTok{:}\DecValTok{50}\NormalTok{, }\KeywordTok{c}\NormalTok{(}\DecValTok{1}\NormalTok{, }\DecValTok{2}\NormalTok{, }\DecValTok{4}\NormalTok{)]}
\end{Highlighting}
\end{Shaded}

\begin{verbatim}
##    atomic_number symbol
## 41            41     Nb
## 42            42     Mo
## 43            43     Tc
## 44            44     Ru
## 45            45     Rh
## 46            46     Pd
## 47            47     Ag
## 48            48     Cd
## 49            49     In
## 50            50     Sn
##                                                                      name_origin
## 41                         Niobe, daughter of king Tantalus from Greek mythology
## 42                                             the Greek molybdos meaning 'lead'
## 43                                     the Greek tekhn??tos meaning 'artificial'
## 44                                       Ruthenia, the New Latin name for Russia
## 45                                     the Greek rhodos, meaning 'rose coloured'
## 46 the then recently discovered asteroid Pallas, considered a planet at the time
## 47                                              English word (argentum in Latin)
## 48                                        the New Latin cadmia, from King Kadmos
## 49                                                                        indigo
## 50                                               English word (stannum in Latin)
\end{verbatim}

\hypertarget{using-logicals}{%
\subsection{Using logicals}\label{using-logicals}}

As you've seen, we can specify directly which elements we'd like to select based on the integer values of the indices of the data frame. Another way to select elements is by using a logical vector:

\begin{Shaded}
\begin{Highlighting}[]
\NormalTok{friend_names[}\KeywordTok{c}\NormalTok{(}\OtherTok{TRUE}\NormalTok{, }\OtherTok{FALSE}\NormalTok{, }\OtherTok{TRUE}\NormalTok{, }\OtherTok{FALSE}\NormalTok{)]}
\end{Highlighting}
\end{Shaded}

\begin{verbatim}
## [1] "Bertha" "Alice"
\end{verbatim}

This can be extended to choose specific elements from a data frame based on the values in the ``cells'' of the data frame. A logical vector like the one above (\texttt{c(TRUE,\ FALSE,\ TRUE,\ FALSE)}) can be generated based on our entries:

\begin{Shaded}
\begin{Highlighting}[]
\NormalTok{friend_names }\OperatorTok{==}\StringTok{ "Bertha"}
\end{Highlighting}
\end{Shaded}

\begin{verbatim}
## [1]  TRUE FALSE FALSE FALSE
\end{verbatim}

We see that only the first element in this new vector is set to \texttt{TRUE} because \texttt{"Bertha"} is the first entry in the friend\_names vector. We thus have another way of subsetting that will return only those names that are \texttt{"Bertha"} or \texttt{"Alice"}:

\begin{Shaded}
\begin{Highlighting}[]
\NormalTok{friend_names[friend_names }\OperatorTok\StringTok{ }\KeywordTok{c}\NormalTok{(}\StringTok{"Bertha"}\NormalTok{, }\StringTok{"Alice"}\NormalTok{)]}
\end{Highlighting}
\end{Shaded}

\begin{verbatim}
## [1] "Bertha" "Alice"
\end{verbatim}

The \texttt{\%in\%} operator looks element-wise in the \texttt{friend\_names} vector and then tries to match each entry with the entries in \texttt{c("Bertha",\ "Alice")}.

Now we can think about how to subset an entire data frame using the same sort of creation of two logical vectors (one for rows and one for columns):

\begin{Shaded}
\begin{Highlighting}[]
\NormalTok{periodic_table[ (periodic_table}\OperatorTok{$}\NormalTok{name }\OperatorTok\StringTok{ }\KeywordTok{c}\NormalTok{(}\StringTok{"Hydrogen"}\NormalTok{, }\StringTok{"Oxygen"}\NormalTok{) ),}
        \KeywordTok{c}\NormalTok{(}\StringTok{"atomic_weight"}\NormalTok{, }\StringTok{"state_at_stp"}\NormalTok{)]}
\end{Highlighting}
\end{Shaded}

\begin{verbatim}
##   atomic_weight state_at_stp
## 1      1.008235          Gas
## 8     15.999000          Gas
\end{verbatim}

The extra parentheses around \texttt{periodic\_table\$name\ \%in\%\ c("Hydrogen",\ "Oxygen")} are a good habit to get into as they ensure everything before the comma is used to select specific rows matching that condition. For the columns, we can specify a vector of the column names to focus on only those variables. The resulting table here gives the \texttt{atomic\_weight} and \texttt{state\_at\_stp} for \texttt{"Hydrogen"} and then for \texttt{"Oxygen"}.

There are many more complicated ways to subset a data frame and one can use the \texttt{subset} function built into R, but in my experience, whenever you want to do anything more complicated than what we have done here, it is easier to use the \texttt{filter} and \texttt{select} functions in the \texttt{dplyr} package.

\hypertarget{functions}{%
\section{Functions}\label{functions}}

You might not have noticed, but we have been using \textbf{functions} throughout this entire chapter. The \texttt{seq} command we saw earlier is a function. It expects a few arguments: \texttt{from}, \texttt{to}, \texttt{by}, and a few others that we didn't specify. How do I know this and why didn't we specify them?

Recall that you can look up the help documentation on any function by entering \texttt{?} and the function name in the R console. If we do this for \texttt{seq} with \texttt{?seq}, we are given some examples of what to expect under the \textbf{Usage} section. R allows for function arguments to take on default values and that's what we see:

\begin{Shaded}
\begin{Highlighting}[]
\KeywordTok{seq}\NormalTok{(}\DataTypeTok{from =} \DecValTok{1}\NormalTok{, }\DataTypeTok{to =} \DecValTok{1}\NormalTok{, }\DataTypeTok{by =}\NormalTok{ ((to }\OperatorTok{-}\StringTok{ }\NormalTok{from)}\OperatorTok{/}\NormalTok{(length.out }\OperatorTok{-}\StringTok{ }\DecValTok{1}\NormalTok{)),}
  \DataTypeTok{length.out =} \OtherTok{NULL}\NormalTok{, }\DataTypeTok{along.with =} \OtherTok{NULL}\NormalTok{, ...)}
\end{Highlighting}
\end{Shaded}

By default, the sequence will both start and end at 1 (a not very interesting sequence). The \texttt{length.out} and \texttt{along.with} arguments are set to \texttt{NULL} by default. \texttt{NULL} represents an empty object in R, so they are essentially ignored, unless you specify values for them. The \texttt{...} argument is beyond the scope of this book, but you can read more about it and other useful tips about writing function at the NiceRCode page \href{http://nicercode.github.io/guides/functions/}{here}.

Not all functions have all default arguments like \texttt{seq}. The \texttt{mean} function is one such example:

\begin{Shaded}
\begin{Highlighting}[]
\KeywordTok{mean}\NormalTok{()}
\end{Highlighting}
\end{Shaded}

\begin{verbatim}
Error in mean.default() : argument "x" is missing, with no default
Calls: <Anonymous> ... withVisible -> eval -> eval -> mean -> mean.default
Execution halted

Exited with status 1.
\end{verbatim}

Notice that R returns an error here, which is not the case if you don't specify the arguments to \texttt{seq}:

\begin{Shaded}
\begin{Highlighting}[]
\KeywordTok{seq}\NormalTok{()}
\end{Highlighting}
\end{Shaded}

\begin{verbatim}
## [1] 1
\end{verbatim}

To fix the error, we'll need to specify which vector or object we want to compute the mean of. Recall the \texttt{ages\_numeric} vector. We can pass that into the \texttt{mean} function:

\begin{Shaded}
\begin{Highlighting}[]
\NormalTok{ages_numeric}
\end{Highlighting}
\end{Shaded}

\begin{verbatim}
## [1] 25 37 22 30
\end{verbatim}

\begin{Shaded}
\begin{Highlighting}[]
\KeywordTok{mean}\NormalTok{(}\DataTypeTok{x =}\NormalTok{ ages_numeric)}
\end{Highlighting}
\end{Shaded}

\begin{verbatim}
## [1] 28.5
\end{verbatim}

We can also skip specifying the name of the argument, as long as you follow the same order as what is given in \textbf{Usage} in the documentation:

\begin{Shaded}
\begin{Highlighting}[]
\KeywordTok{mean}\NormalTok{(ages_numeric)}
\end{Highlighting}
\end{Shaded}

\begin{verbatim}
## [1] 28.5
\end{verbatim}

We can see that R expects the arguments to be \texttt{x}, then \texttt{trim}, and then \texttt{na.rm}. What happens if we try to specify \texttt{TRUE} for \texttt{na.rm} without specifically saying \texttt{na.rm\ =\ TRUE}?

\begin{Shaded}
\begin{Highlighting}[]
\KeywordTok{mean}\NormalTok{(ages_numeric, }\OtherTok{TRUE}\NormalTok{)}
\end{Highlighting}
\end{Shaded}

\begin{verbatim}
Error in mean.default(ages_numeric, TRUE) :
 'trim' must be numeric of length one
Calls: <Anonymous> ... withVisible -> eval -> eval -> mean -> mean.default
Execution halted

Exited with status 1.
\end{verbatim}

Because \texttt{trim} comes before \texttt{na.rm} in the list of arguments, R assumes whatever we put after the first comma is the argument for \texttt{trim}. However, \texttt{trim} expects a fraction between 0 and 0.5, so R informs us that it doesn't understand. It usually is good practice to enter the name of the argument, an equals sign, and then the value you want the argument to take on. The following is clean and helps with readability:

\begin{Shaded}
\begin{Highlighting}[]
\KeywordTok{mean}\NormalTok{(}\DataTypeTok{x =}\NormalTok{ ages_numeric, }\DataTypeTok{na.rm =} \OtherTok{TRUE}\NormalTok{)}
\end{Highlighting}
\end{Shaded}

\begin{verbatim}
## [1] 28.5
\end{verbatim}

\vspace*{0.2in}

\noindent\textbf{Why do some arguments require quotations and others don't?}\vspace*{0.1in}

As you begin to explore help documentation for different functions, you'll notice that some arguments require quotations around them, while others (like those for \texttt{mean}) don't. This brings us back to the discussion earlier about strings, logicals, and numeric/integer classes.

One example of a function that requires a character string (or vector) is the \texttt{install.packages} function, which is at the heart of R's ability to expand on its built-in functionality by importing \textbf{packages} that include functions, templates, and data written by users of R. If you run \texttt{?install.packages}, you'll see that the first argument \texttt{pkgs} is expected to be a character vector. You'll therefore need to enter the packages you'd like to download and install inside quotes.

Two useful packages that I recommend you install and download are \texttt{"ggplot2"} and \texttt{"dplyr"} (if you are using RStudio Server, hopefully your instructor or server administrator has already done so). The \texttt{install.packages} function has a large number of arguments, with all but the \texttt{pkgs}argument set by default. We'll pick a couple here to specify instead of using the defaults:

\begin{Shaded}
\begin{Highlighting}[]
\KeywordTok{install.packages}\NormalTok{(}\DataTypeTok{pkgs =} \KeywordTok{c}\NormalTok{(}\StringTok{"ggplot2"}\NormalTok{,}\StringTok{"dplyr"}\NormalTok{),}
         \DataTypeTok{repos =} \StringTok{"http://cran.rstudio.org"}\NormalTok{,}
         \DataTypeTok{dependencies =} \OtherTok{TRUE}\NormalTok{,}
         \DataTypeTok{quiet =} \OtherTok{TRUE}\NormalTok{)}
\end{Highlighting}
\end{Shaded}

If you refer back to the help via \texttt{?install.packages}, you will see descriptions that what type of argument is expected:

\begin{itemize}
\tightlist
\item
  \texttt{pkgs} expects a character vector
\item
  \texttt{repos} expects a character vector
\item
  \texttt{dependencies} expects a logical value (\texttt{TRUE} or \texttt{FALSE})
\item
  \texttt{quiet} expects a logical value
\end{itemize}

After you've downloaded the packages, you can load the package into your R environment using the \texttt{library} function:

\begin{Shaded}
\begin{Highlighting}[]
\KeywordTok{library}\NormalTok{(}\StringTok{"ggplot2"}\NormalTok{)}
\KeywordTok{library}\NormalTok{(}\StringTok{"dplyr"}\NormalTok{)}
\end{Highlighting}
\end{Shaded}

\hypertarget{closing-thoughts}{%
\section{Closing thoughts}\label{closing-thoughts}}

There are many more advanced analyses that can be done with R, but this chapter provides a way to get started with R without digging in too much. I encourage you to review this chapter frequently as you learn to use R. Quiz yourself on what a specific command does, and then check to see if you are correct. Breaking R is a pretty hard thing to do. Play around, and try to figure out error messages on your own first for 15 minutes or so. Then, if you are still not sure what is going on, check out some of the help with errors in the next chapter.

\hypertarget{errors}{%
\chapter{Deciphering Common R Errors}\label{errors}}

For references on errors, check out the following two links by Noam Ross \href{https://github.com/noamross/zero-dependency-problems/blob/master/misc/stack-overflow-common-r-errors.md}{here} and David Smith \href{http://blog.revolutionanalytics.com/2015/03/the-most-common-r-error-messages.html}{here}.

\hypertarget{error-could-not-find-function}{%
\section{\texorpdfstring{\texttt{Error:\ could\ not\ find\ function}}{Error: could not find function}}\label{error-could-not-find-function}}

This error usually occurs when a package has not been loaded into R via \texttt{library}, so R does not know where to find the specified function. It's a good habit to use the \texttt{library} functions on all of the packages you will be using in the top R chunk in your R Markdown file, which is usually given the chunk name \texttt{setup}.

\hypertarget{error-object-not-found}{%
\section{\texorpdfstring{\texttt{Error:\ object\ not\ found}}{Error: object not found}}\label{error-object-not-found}}

This error usually occurs when your R Markdown document refers to an object that has not been defined in an R chunk at or before that chunk. You'll frequently see this when you've forgotten to copy code from your R Console sandbox back into a chunk in R Markdown.

\hypertarget{misspellings}{%
\section{Misspellings}\label{misspellings}}

One of the most frustrating errors you can encounter in R is when you misspell the name of an object or function. R is not forgiving on this, and it won't try to automatically figure out what you are referring to. You'll usually be able to quite easily figure out that you made a typo because you'll receive an \texttt{object\ not\ found} error.

Remember that R is also case-sensitive, so if you called an object \texttt{Name} and then try to call it \texttt{name} later on without \texttt{name} being defined, you'll receive an error.

\hypertarget{unmatched-parenthesis}{%
\section{Unmatched parenthesis}\label{unmatched-parenthesis}}

Another common error is forgetting or neglecting to finish a call to a function with a closing \texttt{)}. An example of this follows:

\begin{Shaded}
\begin{Highlighting}[]
\KeywordTok{mean}\NormalTok{(}\DataTypeTok{x =} \KeywordTok{c}\NormalTok{(}\DecValTok{1}\NormalTok{, }\DecValTok{5}\NormalTok{, }\DecValTok{10}\NormalTok{, }\DecValTok{52}\NormalTok{)}
\end{Highlighting}
\end{Shaded}

\begin{verbatim}
Error in parse(text = x, srcfile = src) :
 <text>:2:0: unexpected end of input
1: mean(x = c(1, 5, 10, 52)
  ^
Calls: <Anonymous> ... evaluate -> parse_all -> parse_all.character -> parse
Execution halted

Exited with status 1.
\end{verbatim}

In this case, there needs to be one more parenthesis added at the end of your call to \texttt{mean}:

\begin{Shaded}
\begin{Highlighting}[]
\KeywordTok{mean}\NormalTok{(}\DataTypeTok{x =} \KeywordTok{c}\NormalTok{(}\DecValTok{1}\NormalTok{, }\DecValTok{5}\NormalTok{, }\DecValTok{10}\NormalTok{, }\DecValTok{52}\NormalTok{))}
\end{Highlighting}
\end{Shaded}

\hypertarget{general-guidelines}{%
\section{General guidelines}\label{general-guidelines}}

Try your best to not be intimidated by R errors. Oftentimes, you will find that you are able to understand what they mean by carefully reading over them. When you can't, carefully look over your R Markdown file again. You might also want to clear out all of your R environment and start at the top by running the chunks. Remember to only include what you deem your reader will need to follow your analysis.

Even people who have worked with R and programmed for years still use Google and support websites like Stack Overflow to ask for help with their R errors or when they aren't sure how to do something in R. I think you'll be pleasantly surprised at just how much support is available.

\hypertarget{conclusion}{%
\chapter{Concluding Remarks}\label{conclusion}}

My hope is that this book provides a nice introduction into understanding how statisticians, data scientists, and many other professionals use RStudio and R Markdown to simplify their analyses and ensure that their reports are computationally reproducible. Learning R is not nearly as intimidating as it once was, and more and more industries are shifting towards free open-source tools like R and RStudio. R Markdown provides an excellent way to document your analyses and share it with others in a variety of formats.

Of course, this book is just the tip of the iceberg in terms of showing you what R can really do. If you'd like to learn more, I encourage you to check out \href{http://www.moderndive.com/}{Modern Dive}, a free, online, open-source book Albert Kim and I wrote on using modern data analysis techniques and visualization with R, RStudio, and R Markdown.

Additionally, Garrett Grolemund's ``Hands-On Programming with R'' \citep{handson2014} is an excellent resource and goes into much more depth than I do here on how to work with more complicated objects in R. It also discusses concepts in a project-based framework that is entertaining and easy-to-read.

As always, feel free to send me an email at \href{mailto:chester.ismay@gmail.com}{\nolinkurl{chester.ismay@gmail.com}} if you'd like any further clarification or if you have suggestions on improvements. Thanks for taking the time to read through this and best wishes to you on your next steps towards reproducible, thoughtful, beautiful analyses!

\texttt{-} Chester

\renewcommand{\bibname}{References}
\addcontentsline{toc}{chapter}{References}
\bibliography{bib/books.bib,bib/packages.bib,bib/articles.bib}
% 
% 

\end{document}
