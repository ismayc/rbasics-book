\documentclass[]{tufte-book}

% ams
\usepackage{amssymb,amsmath}

\usepackage{ifxetex,ifluatex}
\usepackage{fixltx2e} % provides \textsubscript
\ifnum 0\ifxetex 1\fi\ifluatex 1\fi=0 % if pdftex
  \usepackage[T1]{fontenc}
  \usepackage[utf8]{inputenc}
\else % if luatex or xelatex
  \makeatletter
  \@ifpackageloaded{fontspec}{}{\usepackage{fontspec}}
  \makeatother
  \defaultfontfeatures{Ligatures=TeX,Scale=MatchLowercase}
  \makeatletter
  \@ifpackageloaded{soul}{
     \renewcommand\allcapsspacing[1]{{\addfontfeature{LetterSpace=15}#1}}
     \renewcommand\smallcapsspacing[1]{{\addfontfeature{LetterSpace=10}#1}}
   }{}
  \makeatother
\fi

% graphix
\usepackage{graphicx}
\setkeys{Gin}{width=\linewidth,totalheight=\textheight,keepaspectratio}

% booktabs
\usepackage{booktabs}

% url
\usepackage{url}

% hyperref
\usepackage{hyperref}

% units.
\usepackage{units}


\setcounter{secnumdepth}{2}

% citations
\usepackage{natbib}
\bibliographystyle{apalike}

% pandoc syntax highlighting

% longtable
\usepackage{longtable,booktabs}

% multiplecol
\usepackage{multicol}

% strikeout
\usepackage[normalem]{ulem}

% morefloats
\usepackage{morefloats}


% tightlist macro required by pandoc >= 1.14
\providecommand{\tightlist}{%
  \setlength{\itemsep}{0pt}\setlength{\parskip}{0pt}}

% title / author / date
\title{Getting used to R, RStudio, and RMarkdown}
\author{Chester Ismay}
\date{2016-08-13}

\usepackage{booktabs}
\usepackage{longtable}
\usepackage{framed,color}
\definecolor{shadecolor}{RGB}{248,248,248}

\ifxetex
  \usepackage{letltxmacro}
  \setlength{\XeTeXLinkMargin}{1pt}
  \LetLtxMacro\SavedIncludeGraphics\includegraphics
  \def\includegraphics#1#{% #1 catches optional stuff (star/opt. arg.)
    \IncludeGraphicsAux{#1}%
  }%
  \newcommand*{\IncludeGraphicsAux}[2]{%
    \XeTeXLinkBox{%
      \SavedIncludeGraphics#1{#2}%
    }%
  }%
\fi

%% Need to clean up
\newenvironment{rmdblock}[1]
  {\begin{shaded*}
  \begin{itemize}
  \renewcommand{\labelitemi}{
    \raisebox{-.7\height}[0pt][0pt]{
  %    {\setkeys{Gin}{width=3em,keepaspectratio}\includegraphics{images/#1}}
    }
  }
  \item
  }
  {
  \end{itemize}
  \end{shaded*}
  }
%% Probably can be omitted
\newenvironment{rmdnote}
  {\begin{rmdblock}{note}}
  {\end{rmdblock}}
\newenvironment{rmdcaution}
  {\begin{rmdblock}{caution}}
  {\end{rmdblock}}
\newenvironment{rmdimportant}
  {\begin{rmdblock}{important}}
  {\end{rmdblock}}
\newenvironment{rmdtip}
  {\begin{rmdblock}{tip}}
  {\end{rmdblock}}
\newenvironment{rmdwarning}
  {\begin{rmdblock}{warning}}
  {\end{rmdblock}}
\newenvironment{learncheck}
  {\begin{rmdblock}{warning}}
  {\end{rmdblock}}
\newenvironment{review}
  {\begin{rmdblock}{warning}}
  {\end{rmdblock}}

% To tweak tufte layout
\geometry{
  left=0.8in, % left margin
  textwidth=35pc, % main text block
  marginparsep=1pc, % gutter between main text block and margin notes
  marginparwidth=8pc % width of margin notes
}

\begin{document}

\maketitle



{
\setcounter{tocdepth}{1}
\tableofcontents
}

\chapter{Introduction}\label{intro}

This book was written using the \textbf{bookdown} R package from Yihui
Xie. You can find different formats for the book by clicking on the save
icon \includegraphics[width=0.19in]{screenshots/save_icon} in the top
pane of this book website. HTML is the preferred format but PDF and ePub
formats are also available.

This resource is designed to provide new users to R, RStudio, and
RMarkdown with the introductory steps needed to begin their own
reproducible research. Many screenshots and GIFs will be included, but
if further clarification is needed on these or any other aspect of the
book, please create a GitHub issue
\href{https://github.com/ismayc/rbasics/issues}{here} or email
\href{mailto:chester.ismay@gmail.com}{me} with a reference to the
error/area where more guidance is necessary.

\chapter{R and RStudio Basics}\label{rstudiobasics}

If you are brand new to R and programming, you may be scared. You aren't
used to having to type commands to tell the computer what to do. You may
be more used to using drop-down menus and other graphical user
interfaces that allow you to pick what you'd like to do. So why are so
many companies, colleges/universities, and individuals of all
disciplinary backgrounds shifting towards using R?

There are lots of answers to this question, but I believe the most
important are:

\begin{enumerate}
\def\labelenumi{\arabic{enumi}.}
\item
  R is free. RStudio is free.

  One of the biggest perks about working with R and RStudio is that they
  are both provided free of charge to use. R is an open-source
  programming language that has grown tremendously in recent years with
  developers adding more functionality and packages on a daily basis.
  Where other more proprietary packages are sometimes stuck in the dark
  ages (the 1990s, for example) of development and can be incredibly
  expensive to purchase, R continues to be a free alternative that
  allows users of all levels to contribute.

  RStudio is a graphical user interface that allows one to write R code
  and view the results of that code in an easy way. It is also free to
  download and work with.
\item
  Analyses done using R are reproducible.

  As many scientific fields push towards more reproducible analyses, the
  point-and-click proprietary systems actually serve as a hindrance to
  this process. If you need to rerun your analysis using these systems,
  you'll need to carefully copy-and-paste your analysis into your text
  editors from potentially beginning to end. Anyone that has done this
  sort of copy-and-pasting knows that it is prone to errors and tedious.
\item
  Using R makes collaboration easier.

  It would be much better to be able to update your code/data inputs and
  rerun all of your analysis. Reproducibility also helps you as a
  programmer since your greatest collaborator will probably be yourself
  a few months or years down the road. Instead of having to carefully
  write down all the steps you took to find the right drop-down menu
  option, your entire code is stored.

  This also helps you with collaboration since, as you will see later,
  you can share an RMarkdown file containing all of your analysis,
  documentation, commentary, and the code to others. This reduces the
  time to needed to work with others and reduces the likelihood of
  errors being made in following along with point-and-click analyses.
\item
  Finding answers to questions is much simpler.

  If you have ever had an issue with software, you know how difficult it
  is to find answers to your questions. ``How can I describe the process
  to someone else? Do I need to take screenshots?'' R is a programming
  language and so it is much easier (after a bit of practice) to use
  Google or Stack Overflow to find answers to your questions. I
  frequently (almost on a daily basis) Google things like ``How do I
  make a side-by-side boxplot in R coloring by a third variable?''.
  You'll become better at working with R by reaching out to others for
  help.
\item
  Struggling through programming helps you learn.

  We all know that learning isn't easy. Do you have trouble remembering
  how to follow a list of more than 10 steps or so? Do you find yourself
  going back over and over again because you can't remember what step
  comes next in the process? This is extremely common especially if you
  haven't done the procedure in awhile.

  The unfortunate thing is that our brain tricks us into picking the
  easy route. If you truly want to learn how to do something (like
  programming with R), you'll need to feel frustrated at times. Any time
  you learn something you've been frustrated. (We tend to forget all the
  frustration and only think about where we currently are.) R still
  frustrates me from time to time, but I grow through practice. Hadley
  Wickham encapsulated this phenomenon nicely in the Prologue of the
  book ``Hands-On Programming with R'' \citep{handson2014}:

  \begin{quote}
  As you learn to program, you are going to get frustrated. You are
  learning a new language, and it will take time to become fluent. But
  frustration is not just natural, it's actually a positive sign that
  you should watch for. Frustration is your brain's way of being lazy;
  it's trying to get you to quit and go do something easy or fun. If you
  want to get physically fitter, you need to push your body even though
  it complains. If you want to get better at programming, you'll need to
  push your brain. Recognize when you get frustrated and see it as a
  good thing: you're now stretching yourself. Push yourself a little
  further every day, and you'll soon be a confident programmer.
  \end{quote}
\end{enumerate}

\begin{itemize}
\tightlist
\item
  What is R? What is RStudio?
\item
  Installing R and RStudio directions with screenshots
\item
  Screenshots of RStudio frames?
\end{itemize}

\textbf{Last updated:}

\begin{verbatim}
## [1] "Saturday, August 13, 2016 15:53:34 PDT"
\end{verbatim}

\chapter{R Markdown}\label{rmarkdown}

\begin{itemize}
\tightlist
\item
  Walk through the components of an R Markdown file
\item
  Resource for Markdown:
  \url{https://github.com/adam-p/markdown-here/wiki/Markdown-Cheatsheet}
\item
  RMarkdown chunk options
\item
  Help -\textgreater{} Cheatsheets
\end{itemize}

\chapter{Introductory R analysis using R Markdown}\label{rmdanal}

\begin{itemize}
\tightlist
\item
  ``File organization and naming are powerful weapons against chaos.'' -
  Jenny Bryan
\item
  Give an introduction into using R with periodic table dataset
\item
  Mean, median, standard deviation, five-number summary, distribution
\item
  Some content to cover:

  \begin{itemize}
  \tightlist
  \item
    data structures (vectors, lists, data frames, matrices)
  \item
    indexing/subsetting
  \item
    functions (default arguments)
  \item
    Case matters in R!
  \item
    Why do some arguments require quotations and others don't?
  \end{itemize}
\end{itemize}

\bibliography{bib/packages.bib,bib/books.bib,bib/articles.bib}



\end{document}
